\documentclass[a4paper,11pt,twocolumn]{article}

\usepackage{luatexja}
\usepackage{luatexja-fontspec}
\setmainjfont{IPAexMincho}
\usepackage{graphicx}
\usepackage{here}
\usepackage{url}
\usepackage{cite}
\usepackage{amsmath}

\setlength{\textwidth}{170mm}
\setlength{\oddsidemargin}{-5mm}
\setlength{\textheight}{250mm}
\setlength{\topmargin}{-15mm}

\title{%
BLE電波強度を用いた屋内位置推定手法の精度評価と改善に関する研究\\
\large 
}

\author{%
近畿大学 情報学部 情報学科\\
学籍番号:2312110159\\
氏名:内山敦也
}

\date{}

\begin{document}
\maketitle

\section{序論}

近年、スマートフォンや IoT デバイスの普及に伴い、屋内環境における位置情報サービスの需要が高まっている。
屋外では GPS による位置推定が一般的である一方、建物内部では電波遮蔽により測位精度が大きく低下するため、代替手法として
Wi-Fi、BLE(Bluetooth Low Energy)、UWB などの無線技術を用いた屋内測位手法が注目されている。

その中でも BLE は省電力・低コスト・スマートフォン標準搭載という利点から、実運用に適した技術として広く研究が進められている。
しかし、BLE 電波強度(RSSI: Received Signal Strength Indicator)は外乱の影響を強く受け、瞬間的なノイズや反射により
値が大きく変動するという問題がある。そのため、RSSI を距離に変換して位置推定に利用する手法では、
外れ値によって数 m ~ 数十 m の誤差が生じ、安定した測位精度を確保することが困難である。

本研究の目的は以下の 2 点である。
\begin{itemize}
    \item BLE RSSI のノイズ特性を考慮した平滑化手法を設計・実装し、測位精度を改善すること
    \item 実測データに基づき、平滑化処理の有無による位置推定精度を比較評価すること
\end{itemize}

本稿では、BLE 測位システムの構築方法、平滑化アルゴリズム、評価実験の手法および結果について報告し、
今後の改善点についても議論する。

\section{研究内容:BLEを用いた屋内位置推定手法}
本研究では、工場環境において BLE(Bluetooth Low Energy)タグから発信される電波強度(RSSI)を用いた屋内測位手法の精度向上を目的とし、M5Stack/ESP32デバイスを複数設置した受信システムを構築した。各受信機が取得した RSSI をフィルタリングし、三点測量法(Trilateration)によって作業員タグの位置推定を行う。また、工場の金属棚や機械による電波反射・遮蔽の影響を評価するため、受信機台数の増加が推定精度に与える影響についても検証する。さらに、複数台の M5 デバイス間の時刻同期方法、RSSI の変動に対する平滑化手法、推定値のログ収集・可視化システムの構築を通じて、実運用可能な位置推定基盤の実現性を検討する。

\section{今後の計画}

\section{参考文献}
\bibliographystyle{unsrt}
\bibliography{refs}

\end{document}

\section{研究内容:BLEを用いた屋内位置推定手法}
BLEを用いた屋内測位の精度改善については Shuang ら \cite{shuang2017ipin} によって多点測位手法が提案されている。


\section{今後の計画}


\section{参考文献}
\bibliographystyle{unsrt}
\bibliography{refs}

\end{document}
